\documentclass{article}
\usepackage[margin=2cm]{geometry}
\usepackage{hyperref}
\usepackage{longtable}
\title{Flexible Snow Model user guide}
\author{Richard Essery}
\date{2 January 2019}
\begin{document}
\maketitle
\parindent0pt

\section{FSM2}

The Flexible Snow Model (FSM2) is a multi-physics energy balance model of snow accumulation and melt, extending the Factorial Snow Model (FSM) with additional physics, driving and output options. FSM2 adds forest canopy model options and the possibility of running simulations for more than one point at the same time. For greater efficiency in long simulations than FSM, which selects physics options when it is run, FSM2 options are selected when the model is compiled. Otherwise, FSM2 is built and run in the same way as FSM.

\section{Compiling the model}

FSM2 is coded in Fortran and consists of subroutines and modules contained in the {\tt src} directory. A linux executable {\tt FSM2} or a Windows executable {\tt FSM2.exe} is produced by running script {\tt compil.sh} or batch file {\tt compil.bat}. Both use the gfortran compiler but could be edited for other compilers. Driving data, physics and output configurations are selected in the compilation script by defining option numbers that are copied to a preprocessor file {\tt src/OPTS.h} before compilation.

\subsection{Driving data compilation options }
\begin{longtable}{|l|l|l|}
\hline
Option number & Description & Options \\
\hline
{\tt DRIV1D} & 1D driving data format
&   0 - FSM format \\
& & 1 - extended data format with SW components\\
& & 2 - \href{https://www.geos.ed.ac.uk/~ressery/ESM-SnowMIP/ESMSnowMIP_Reference_sites.pdf}{ESM-SnowMIP} format \\
\hline
{\tt DOWNSC} & 1D driving data downscaling
&   0 - no \\
& & 1 - yes \\
\hline
{\tt DEMHDR} & Header information in DEM for downscaling
&   0 - none \\
& & 1 - ESRI format \\
\hline
{\tt SWPART} & Shortwave radiation partition
&   0 - Total SW radiation used \\
& & 1 - Diffuse and direct SW calculated \\
& & 2 - Diffuse and direct SW in extended data \\
\hline
{\tt ZOFFST} & Measurement height offset
&   0 - Height above ground \\
& & 1 - Height above canopy top \\
\hline
\end{longtable}

\subsection{Physics compilation options }
\begin{longtable}{|l|l|l|}
\hline
Option number & Description & Options   \\
\hline
{\tt ALBEDO} & Snow albedo
&   0 - diagnostic temperature function \\
& & 1 - prognostic age function         \\
\hline
{\tt CANMOD} & Forest canopy
&   0 - zero layer                      \\
& & 1 - one layer                       \\
\hline
{\tt CONDCT} & Thermal conductivity of snow
&   0 - fixed                           \\
& & 1 - function of density             \\
\hline
{\tt DENSTY} & Snow density
&   0 - fixed                           \\
& & 1 - function of age                 \\
& & 2 - function of overburden          \\
\hline
{\tt EXCHNG} & Surface-atmosphere exchange
&   0 - fixed exchange coefficient      \\
& & 1 - function of Richardson number   \\
\hline
{\tt HYDROL} & Snow hydrology
&   0 - free draining                   \\
& & 1 - bucket                          \\ 
\hline
{\tt SNFRAC} & Snow cover fraction
&   0 - $f_s = h/(h+h_f)$               \\
& & 1 - $f_s = \tanh(h/h_f)$            \\ 
\hline
\end{longtable}

\section{Running the model}

FSM2 requires meteorological driving data and namelists to set options and parameters. The model is run with the commands {\tt ./FSM2 < nlst.txt} or {\tt FSM2.exe < nlst.txt}, where {\tt nlst.txt} is a text file containing eight namelists described in tables below. All of the namelists have to be present in the order of the tables, but any or all of the variables listed in a namelist can be omitted; defaults are then used.


\subsection{Grid dimensions namelist {\tt \&gridpnts}}
FSM2 can be run at a point, at a sequence of points, with a range of surface characteristics or on a rectangular grid by selecting values for dimensions {\tt Nx} and {\tt Ny}. 

\begin{longtable}{|l|l|l|}
\hline
Variable & Default & Description \\
\hline
{\tt Nsmax}      & 3    & Maximum number of snow layers                       \\
{\tt Nsoil}      & 4    & Number of soil layers                               \\
{\tt Nx}         & 1    & Number of grid points in x direction or in sequence \\
{\tt Ny}         & 1    & Number of grid points in y direction                \\
{\tt ztop\_file} & none & DEM file name                                       \\
\hline
\end{longtable}

A DEM file has to be specified if FSM2 is complied with {\tt DOWNSC=1}. Files in the \href{http://resources.esri.com/help/9.3/arcgisdesktop/com/gp_toolref/spatial_analyst_tools/esri_ascii_raster_format.htm}{ESRI ASCII raster format} ({\tt DEMHDR=1}) have six header lines with grid information, e.g.
{\tt\begin{verbatim}
ncols         1000
nrows         1000
xllcorner     215000
yllcorner     770000
cellsize      5
NODATA_value  -9999
\end{verbatim}}
If provided, {\tt ncols} and {\tt nrows} overwrite {\tt Nx} and {\tt Ny} from {\tt \&gridpnts}.

\subsection{Model levels namelist {\tt \&gridlevs}}
Snow and soil layers are numbered and listed from the top downwards. If layer thicknesses are specified in {\tt \&gridlevs}, they must match the numbers of layers specified in {\tt \&gridpnts}.

\begin{longtable}{|l|l|l|}
\hline
Variable     & Default                     & Description           \\
\hline
{\tt Dzsnow} & 0.1, 0.2, snowdepth - 0.3 m & Snow layer thicknesses \\
{\tt Dzsoil} & 0.1, 0.2, 0.4, 0.8 m        & Soil layer thicknesses \\
\hline
\end{longtable}

\subsection{Driving data namelist {\tt \&drive} and driving data files}

\begin{longtable}{|l|l|l|l|}
\hline
Variable        & Default & Description                               & Used by       \\
\hline
{\tt met\_file} & 'met'   & Driving data file name                    &               \\
{\tt dt}        & 3600 s  & Timestep                                  &               \\
{\tt zT}        & 2 m     & Temperature and humidity measurement height &             \\
{\tt zU}        & 10 m    & Wind speed measurement height             &               \\
{\tt lat}       &0$^\circ$& Latitude                                  &{\tt SWPART=1} \\
{\tt noon}      & 12.00   & Time of solar noon                        &{\tt SWPART=1} \\
{\tt Pscl}      & 0.35 km$^{-1}$  & Precipitation adjustment scale    &{\tt DOWNSC=1} \\
{\tt Tlps}      & 6.5 K km$^{-1}$ & Temperature lapse rate            &{\tt DOWNSC=1} \\
{\tt Tsnw}      & 2$^\circ$C      & Snow threshold temperature        &{\tt DOWNSC=1} \\
{\tt zaws}      & 0 m     & Weather station elevation for downscaling &{\tt DOWNSC=1} \\
\hline 
\end{longtable}

Measurement heights are specified above the ground if FSM2 is compiled with {\tt ZOFFST=0} and above the canopy top if {\tt ZOFFST=1} (required for driving with reanalyses). For simulations at a point or for a set of nearby points with common meteorology, 1D driving data are read from the named text file. Driving variables are arranged in columns of the file and rows correspond with timesteps.

\begin{longtable}{|l|l|l|l|}
\hline
Variable    & Units    & Description                  & Used by \\
\hline
{\tt year}  & years    & Year                         &\\
{\tt month} & months   & Month of the year            &\\
{\tt day}   & days     & Day of the month             &\\
{\tt hour}  & years    & Hour of the day              &\\
{\tt LW} & W m$^{-2}$  & Incoming longwave radiation  &\\
{\tt Ps} & Pa          & Surface air pressure         &\\
{\tt Rf} & kg m$^{-2}$ s$^{-1}$ & Rainfall rate       &\\
{\tt RH} & \%          & Relative humidity            &\\
{\tt Sf} & kg m$^{-2}$ s$^{-1}$ & Snowfall rate       &\\
{\tt Ta} & K           & Air temperature              &\\
{\tt Ua} & m s$^{-1}$  & Wind speed                   &\\
{\tt SW} & W m$^{-2}$  & Incoming shortwave radiation & {\tt DRIV1D=0,2}   \\
{\tt Sdif} &W m$^{-2}$ & Diffuse shortwave radiation  & {\tt DRIV1D=1}     \\ 
{\tt Sdir} &W m$^{-2}$ & Direct-beam shortwave radiation & {\tt DRIV1D=1}  \\ 
%{\tt Udir} & $^\circ$  & Wind direction, clockwise from N & {\tt DRIV1D=1} \\
{\tt Qa} & kg kg$^{-1}$& Specific humidity            & {\tt DRIV1D=2}     \\ 
\hline 
\end{longtable}

The columns in a 1D driving data file are:

\begin{tabular}{ll}
{\obeyspaces\tt{year month day hour SW   LW   Sf Rf Ta RH Ua Ps}}    &for {\tt DRIV1D=0} \\
{\tt year month day hour Sdif Sdir LW Sf Rf Ta RH Ua Ps}             &for {\tt DRIV1D=1} \\
{\obeyspaces\tt{year month day hour SW   LW   Rf Sf Ta Qa RH Ua Ps}} &for {\tt DRIV1D=2} \\
\end{tabular}

\subsection{Parameters namelist {\tt \&params}}

The parameters used depend on which options are selected and whether a forest canopy is specified.

\begin{longtable}{|l|l|l|l|}
\hline
Parameter & Default & Description & Used by \\
\hline
\hline \multicolumn{4}{|c|}{Snow albedo parameters} \\
\hline 
{\tt asmx} & 0.8            & Maximum albedo for fresh snow           &               \\
{\tt asmn} & 0.5            & Minimum albedo for melting snow         &               \\
{\tt hfsn} & 0.1 m          & Snow cover fraction depth scale         &               \\
{\tt Talb} & -2$^\circ$C    & Snow albedo decay temperature threshold &{\tt ALBEDO=0} \\
{\tt Salb} & 10 kg m$^{-2}$ & Snowfall to refresh albedo              &{\tt ALBEDO=1} \\
{\tt tcld} & 1000 h         & Cold snow albedo decay time scale       &{\tt ALBEDO=1} \\
{\tt tcld} & 100 h          & Melting snow albedo decay time scale    &{\tt ALBEDO=1} \\
{\tt rgr0} & $5\times10^{-5}$ m & Fresh snow grain radius             &               \\
\hline 
\hline 
\multicolumn{4}{|c|}{Snow thermal conductivity parameters} \\
\hline 
{\tt kfix} & 0.24 W m$^{-1}$ K$^{-1}$ & Fixed thermal conductivity    &{\tt CONDCT=0} \\
{\tt bthr} & 2                        & Thermal conductivity exponent &{\tt CONDCT=1} \\
\hline
\hline
\multicolumn{4}{|c|}{Snow density parameters} \\
\hline 
{\tt rho0} & 300 kg m$^{-3}$ & Fixed snow density                     &{\tt DENSTY=0}   \\
{\tt rhof} & 100 kg m$^{-3}$ & Fresh snow density                     &{\tt DENSTY=1,2} \\
{\tt rcld} & 300 kg m$^{-3}$ & Maximum density for cold snow          &{\tt DENSTY=1}   \\
{\tt rmlt} & 500 kg m$^{-3}$ & Maximum density for melting snow       &{\tt DENSTY=1}   \\
{\tt trho} & 200 h           & Snow compaction time scale             &{\tt DENSTY=1}   \\
{\tt eta0} & $3.7 \times 10^7$ Pa s  & Reference snow viscosity       &{\tt DENSTY=2}   \\
{\tt snda} & $2.8 \times 10^{-6}$ s$^{-1}$ & Thermal metamorphism parameter &{\tt DENSTY=2} \\
\hline
\hline 
\multicolumn{4}{|c|}{Turbulent exchange parameters} \\
\hline 
{\tt z0sn} & 0.01 m          & Snow surface roughness length          &               \\
{\tt bstb} & 5               & Atmospheric stability parameter        &{\tt EXCHNG=1} \\
\hline
\hline 
\multicolumn{4}{|c|}{Snow hydraulics parameters}                                      \\
\hline
{\tt Wirr} & 0.03          & Irreducible liquid water content of snow &{\tt HYDROL=1} \\
\hline
\hline 
\multicolumn{4}{|c|}{Soil parameters}                                                 \\
\hline 
{\tt gsat} & 0.01 m s$^{-1}$ & Surface conductance for saturated soil &               \\
\hline 
\hline 
\multicolumn{4}{|c|}{Solver parameters}                                               \\
\hline 
{\tt Nitr} & 4      & Number of iterations in energy balance calculation &            \\
\hline 
\end{longtable}

\begin{longtable}{|l|l|l|}
\hline
\multicolumn{3}{|c|}{Canopy parameters} \\
\hline
Parameter & Default & Description \\
\hline
{\tt avg0} & 0.1             & Snow-free vegetation albedo                  \\
{\tt avgs} & 0.4             & Snow-covered vegetation albedo               \\ 
{\tt cden} & 0.004           & Dense canopy turbulent transfer coefficient  \\
{\tt cvai} & 4.4 kg m$^{-2}$ & Canopy snow capacity per unit VAI            \\
{\tt cveg} & 20              & Vegetation turbulent transfer coefficient    \\
{\tt Gcn1} & 0.5             & Leaf angle distribution parameter            \\
{\tt Gcn2} & 0               & Leaf angle distribution parameter            \\
{\tt gsnf} & 0 m s$^{-1}$    & Snow-free vegetation moisture conductance    \\
{\tt kdif} & 0.5             & Diffuse radiation extinction coefficient     \\
{\tt kveg} & 1.0             & Canopy cover coefficient                     \\
{\tt rchd} & 0.67            & Displacement height to canopy height ratio   \\
{\tt rchz} & 0.1             & Roughness length to canopy height ratio      \\
{\tt tcnc} & 240 h           & Canopy unloading time scale for cold snow    \\
{\tt tcnm} & 2.4 h           & Canopy unloading time scale for melting snow \\
\hline 
\end{longtable}

\subsection{Site characteristics namelist {\tt \&maps} and map files}

\begin{longtable}{|l|l|l|}
\hline
Parameter & Default & Description \\
\hline
{\tt alb0}  & 0.2             & Snow-free ground albedo                    \\
{\tt canh}  & 2500 {\tt VAI}  & Canopy heat capacity (J K$^{-1}$ m$^{-2}$) \\
{\tt fcly}  & 0.3             & Soil clay fraction                         \\
{\tt fsnd}  & 0.6             & Soil sand fraction                         \\
{\tt fsky}  & 1               & Sky view fraction                          \\
{\tt fveg}  & {\tt 1 - exp(-kveg VAI)}  & Canopy cover fraction            \\
{\tt hcan}  & 0               & Canopy height (m)                          \\
{\tt trcn}  & {\tt exp(-kdif VAI)}  & Canopy transmissivity                \\
{\tt VAI}   & 0               & Vegetation area index                      \\
{\tt z0sf}  & 0.1             & Snow-free ground roughness length          \\
\hline 
\end{longtable}

Site characteristics can either be left as default values, set to a sequence of {\tt Nx$\times$Ny} values in the namelist or read from a named map file. e.g. for a simulation with 10 points, the snow-free ground albedo can be reset to a constant value of 0.1 in {\tt \&maps} by including

{\tt alb0 = 10*0.1}

or set to a sequence (with spaces or commas) by including

{\tt alb0 = 0.2 0.2 0.1 0.1 0.1 0.1 0.1 0.1 0.2 0.2} 

or read from a file {\tt albedo.txt} containing 10 values by including

{\tt alb0\_file = 'albedo.txt'}

Sky view can be set independently of vegetation cover to allow for grid cells shaded by topography or vegetation in neighbouring cells.

\subsection{Initial values namelist {\tt \&initial} and start files}

\begin{longtable}{|l|l|l|}
\hline
Variable & Default & Description \\
\hline
{\tt start\_file}  & {\tt 'none'}  & Start file name                                      \\
{\tt fsat}  & 4*0.5  & Initial moisture content of soil layers as fractions of saturation \\
{\tt Tsoil} & 4*285  & Initial temperature of soil layers                                 \\

\hline 
\end{longtable}

Soil temperature and moisture content are taken from the namelist and FSM2 is initialized in a snow-free state by default. If a start file is named, it should be a text file containing initial values for each of the state variables in order:

\begin{longtable}{|l|l|l|}
\hline
Variable & Units & Description \\
\hline
{\tt albs(Nx,Ny)}        & -            & Albedo of snow                             \\
{\tt Ds(Nsmax,Nx,Ny)}    & m            & Thickness of snow layers                   \\
{\tt Nsnow(Nx,Ny)}       & -            & Number of snow layers                      \\
{\tt Qcan(Nx,Ny)}        & kg kg$^{-1}$ & Canopy air space specific humidity         \\
{\tt rgrn(Nsmax,Nx,Ny)}  & m            & Snow grain radii in layers                 \\
{\tt Sice(Nsmax,Nx,Ny)}  & kg m$^{-2}$  & Ice content of snow layers                 \\
{\tt Sliq(Nsmax,Nx,Ny)}  & kg m$^{-2}$  & Liquid content of snow layers              \\
{\tt Sveg(Nx,Ny)}        & W m$^{-2}$   & Snow mass on vegetation                    \\
{\tt Tcan(Nx,Ny)}        & K            & Canopy air space temperature               \\
{\tt theta(Nsoil,Nx,Ny)} & -            & Volumetric moisture content of soil layers \\
{\tt Tsnow(Nsmax,Nx,Ny)} & K            & Temperature of snow layers                 \\
{\tt Tsoil(Nsoil,Nx,Ny)} & K            & Temperature of soil layers                 \\
{\tt Tsrf(Nx,Ny)}        & K            & Surface skin temperature                   \\
{\tt Tveg(Nx,Ny)}        & K            & Vegetation temperature                     \\
\hline 
\end{longtable}

The easiest way to generate a start file is to spin up the model by running it for a whole number of years without a start file and then rename the dump file produced at the end of the run as a start file for a new run.

\subsection{Output namelist {\tt \&outputs} and text output files}

\begin{longtable}{|l|l|l|}
\hline
Variable & Default & Description \\
\hline
{\tt Nave}       & 24           & Number of timesteps in averaged outputs       \\
{\tt Nsmp}       & 12           & Timestep of sample outputs, $\leq$ {\tt Nave} \\
{\tt runid}      & none         & Run identifier string                         \\
{\tt dump\_file} & {\tt 'dump'} & Dump file name                                \\
\hline 
\end{longtable}

A run identifier, if specified, is prefixed on all output file names. If the run identifier includes a directory name (e.g. {\tt runid = 'output/'}), the directory has to exist before the model is run. A metadata file {\tt runid+'runifo'} containing copies of all the namelists and compilation options is written at the start of a run, and the state variables are written to a dump file {\tt runid+dump\_file} with the same format as the start file at the end of a run. There are two options for plain text outputs

Flux variable are averaged over {\tt Nave} timesteps and written to file {\tt runid+'ave'}, and state variables are written to file {\tt runid+'smp'} at timestep number {\tt Nsmp} during every averaging period. For the default output frequencies, daily averages and samples at noon will be produced if the driving data has a one-hour timestep and starts at 01:00. Full timeseries are written if {\tt Nave=1} and {\tt Nsmp=1}. 

The sample file has 4 + 6$\times${\tt Nx$\times$Ny} columns:

\begin{longtable}{|l|l|l|}
\hline
Variable & Units & Description \\
\hline
{\tt year}        & years       & Year                     \\
{\tt month}       & months      & Month of the year        \\
{\tt day}         & days        & Day of the month         \\
{\tt hour}        & hours       & Hour of the day          \\
{\tt snd(Nx*Ny)}  & m           & Snow depth               \\
{\tt SWE(Nx*Ny)}  & kg m$^{-2}$ & Snow water equivalent    \\
{\tt Sveg(Nx*Ny)} & kg m$^{-2}$ & Snow mass on vegetation  \\
{\tt Tsrf(Nx*Ny)} & K           & Surface temperature      \\
{\tt Tsoil(Nx*Ny)}& K           & Level 2 soil temperature \\
{\tt Tveg(Nx*Ny)} & K           & Vegetation temperature   \\
\hline 
\end{longtable}

The average file has 4 + 11$\times${\tt Nx$\times$Ny} columns:

\begin{longtable}{|l|l|l|}
\hline
Variable & Units & Description \\
\hline
{\tt year}        & years       & Year                                 \\
{\tt month}       & months      & Month of the year                    \\
{\tt day}         & days        & Day of the month                     \\
{\tt hour}        & hours       & Hour of the day                      \\
{\tt alb(Nx*Ny)}  & -           & Flux-weighted albedo                 \\
{\tt G(Nx*Ny)}    & W m$^{-2}$  & Ground heat flux                     \\
{\tt Gsoil(Nx*Ny)}& W m$^{-2}$  & Heat flux into soil                  \\
{\tt H(Nx*Ny)}    & W m$^{-2}$  & Sensible heat flux to the atmosphere \\
{\tt Hsrf(Nx*Ny)} & W m$^{-2}$  & Sensible heat flux from the surface  \\
{\tt LE(Nx*Ny)}   & W m$^{-2}$  & Latent heat flux to the atmosphere   \\
{\tt LEsrf(Nx*Ny)}& W m$^{-2}$  & Latent heat flux from the surface    \\
{\tt Melt(Nx*Ny)} & kg m$^{-2}$ & Cumulated melt                       \\
{\tt Rnet(Nx*Ny)} & W m$^{-2}$  & Net radiation                        \\
{\tt Roff(Nx*Ny)} & kg m$^{-2}$ & Cumulated runoff                     \\
{\tt Rsrf(Nx*Ny)} & W m$^{-2}$  & Net radiation absorbed by the surface\\
\hline 
\end{longtable}

For simulations at a single point ({\tt Nx=Ny=1}) vertical profiles of snow and soil state variables are written to file {\tt runid+'prf'}. For each timestep in the profile output file, there is a line

\hskip20pt{\tt year month day hour Nsnow Nsoil}

followed by {\tt Nsnow} lines

\hskip20pt{\tt zl Ds rgrain Sice Sliq Tsnow}

and then {\tt Nsoil} lines

\hskip20pt{\tt zl Dzsoil theta Tsoil},

where {\tt zl} is the height of the middle of a layer above the ground (negative for soil layers).


\end{document}

